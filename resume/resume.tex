\documentclass[margin,line,pifont,palatino,courier]{res}


\usepackage{longtable}
\usepackage{pifont}
\usepackage[latin1] {inputenc}

\topmargin=-0.45in
\evensidemargin=-.25in
\oddsidemargin=-.25in
\textwidth=5.5in
\textheight=10.0in
\headsep=0.25in

%\topmargin .5in
%\oddsidemargin -.5in
%\evensidemargin -.5in
%\textwidth=6.0in
%\textheight=9.0in
%\itemsep=0in
%\parsep=0in
\usepackage{fancyhdr}
%\topmargin=0in
%\textheight=8.5in
\pagestyle{fancy}
\renewcommand{\headrulewidth}{0pt}
\fancyhf{}
%\cfoot{\thepage}
%\lfoot{\textit{\footnotesize Research Statement}}
\rfoot{{\footnotesize Curriculum Vitae, Vimal Manohar, \thepage}}


\newenvironment{list1}{
  \begin{list}{\ding{113}}{%
      \setlength{\itemsep}{0in}
      \setlength{\parsep}{0in} \setlength{\parskip}{0in}
      \setlength{\topsep}{0in} \setlength{\partopsep}{0in}
      \setlength{\leftmargin}{0.17in}}}{\end{list}}
\newenvironment{list2}{
  \begin{list}{$\bullet$}{%
      \setlength{\itemsep}{0in}
      \setlength{\parsep}{0in} \setlength{\parskip}{0in}
      \setlength{\topsep}{0in} \setlength{\partopsep}{0in}
      \setlength{\leftmargin}{0.2in}}}{\end{list}}

\begin{document}

\name{Vimal Manohar \vspace*{.1in}}

\begin{resume}

\section{\sc Contact Information}

\vspace{.05in}
\begin{tabular}{l l}
The Center for Language and Speech Processing, \\
Hackerman Hall 322,\\
3400 North Charles Street,                        & \verb+vimal.manohar91@gmail.com+\\
Johns Hopkins University,                  & \verb+http://vimalmanohar.github.io+\\
Baltimore, MD 21218, USA               & \\
\end{tabular}

\section{\sc Research Interests}
Automatic Speech Recognition, Machine Learning, Speech Signal Processing, Natural Language Processing

\section{\sc Education}

\textbf{Johns Hopkins University, Baltimore, MD} \\
Major: Electrical \& Computer Engineering \\
Master of Science in Engineering (M.S.E.), 2015\\
Ph.D., 2018 (Expected) \\
Advisors: Sanjeev Khudanpur and Daniel Povey

\textbf{Indian Institute of Technology Madras, Chennai, India} \\
Major: Electrical Engineering, \quad Minor: Operations Research \\
Bachelor of Technology (B.Tech), 2013 (CGPA: 9.6/10) \\
Advisor: S Umesh, \\

\vspace{-2pt}

\section{\sc Publications}
\begin{itemize}
  \item 
    \textbf{Manohar, V.}; Povey, D.; Khudanpur, S., 
    \textit{``Semi-supervised Maximum Mutual Information Training of Deep Neural
    Network Acoustic Model,"}
    INTERSPEECH 2015. \textbf{Nominated for best students' paper award}.
  \item
    Trmal, J.; \textbf{Manohar, V.} et al., 
    \textit{``A keyword search system using open source software," } 
    Spoken Language Technology Workshop (SLT), 2014 IEEE , vol., no., pp.530,535, 7-10 Dec. 2014
    doi: 10.1109/SLT.2014.7078630 
  \item
    \textbf{Manohar, V.}; Srinivas, C.B.; Umesh, S., 
    \textit{``Acoustic modeling
    using transform-based phone-cluster adaptive training,"} 
    Automatic Speech Recognition and Understanding (ASRU), 2013 IEEE Workshop on
    , vol., no., pp.49,54, 8-12 Dec. 2013 doi: 10.1109/ASRU.2013.6707704
\end{itemize}
\section{\sc Research and Industrial Experience}

\textbf{Jelinek Summer Workshop on Speech and Language Technology (JSALT) 2015 } \\
University of Washington Seattle, Seattle, WAS, USA \hfill 
July -- August '15 \vspace{2pt} \\
Member of the research group working on ``Probabilitic Transcription of Languages with no native-language transcribers''. We showed the utility of mismatched transcriptions from non-native crowdworkers for ASR. (submitted to ICASSP, 2016)
\vspace{-5pt}

\textbf{Research Assistant at The Center for Language and Speech Processing} \\
Johns Hopkins University, Baltimore, MD, USA \hfill Aug '13 -- Present \vspace{2pt} \\
\textit{IARPA Babel} \\
Developed acoustic models for languages in low-resource setting, HMM-GMM based automatic
speech segmentation for ASR, semi-supervised training
approaches for hybrid HMM-DNNs and bottleneck feature NNs \\ 
(published in SLT, 2014). \\
\\
\textit{DARPA BOLT}\\
Developed multilingual-architecture DNN systems for transfer learning from standard Arabic to Egyptian Arabic\\
\vspace{-5pt}

\textbf{Intern at Analog Devices Inc.} \\
Cambridge, MA, USA\hfill May -- Aug '14 \vspace{2pt} \\
Worked on time-frequency masks with multichannel audio for robust speech recognition
\vspace{-5pt}

\textbf{Bachelor's Thesis Project} \\
Indian Institute of Technology Madras, Chennai, India\hfill Sept '12 -- May '13 \vspace{2pt}  \\
Proposed a modification to the HMM-GMM acoustic modeling technique to deal with low-resource settings. We constrained the subspace containing HMM-GMM mean vectors to be defined by piecewise linear transformations of canonical GMM means. \\
(published in ASRU, 2013)\\
\vspace{-5pt}

\textbf{Time-scaling and Pitch-scaling of synthesized speech} \\
Indian Institute of Technology Madras, Chennai, India \hfill March -- May '12 \vspace{2pt} \\
Investigated algorithms for robust VAD, robust pitch estimation, pitch-mark extraction, pitch synchronous overlap-add method of speech synthesis to change the duration and pitch of speech signals
\vspace{-5pt}

\textbf{Research Intern at The Institute of Automation}  \\
University of Bremen, Bremen, Germany\hfill May -- July '12 \vspace{2pt}\\
Implemented a method for estimation of size, position and orientation of
isolated 3D objects using a single pair of stereo images \\
\vspace{-5pt}

\textbf{Texas Instruments Analog Design Contest 2011} \\
Indian Institute of Technology Madras, Chennai, India \hfill Sept '11 -- Feb '12 \vspace{2pt} \\
Designed and constructed a pulse oximeter on an embedded system for real-time estimation of respiratory rate. Among the top 25 entries to the TI India Analog Design Contest 2011.\\
\vspace{-5pt}

\section{\sc Teaching Experience}
\begin{tabular}{@{}p{0.9in} p{4in}}
Fall 2015 & Teaching Assistant, Random Signal Analysis \\
\end{tabular}


\section{\sc Coursework}

\begin{tabular}{@{}p{2.3in}p{3in}}
\begin{list1}
\item Representation learning
\item Speech and audio processing by humans and machines
\item Information Extraction
\item Matrix Analysis
\item Random Signal Analysis

\end{list1}
&
\begin{list1}
\item Speech Technology
\item Compressed Sensing and Sparse Recovery
\item Information Theory
\item Graph Theory
\item Advanced Operations Research
\end{list1}

\end{tabular}

%\section{\sc Extra-\\ Curricular \\ Activities}
%
%    \textbf{Robotics} 
%\begin{itemize} \itemsep -2pt
%        \item Member of the team representing IIT Madras at the National Robotic Contest, Abu Robocon 2011. The team finished among the Top 5 in the country.
%        \item Winner of Autonomous Robotics and Image Processing Robotics competitions held at Techfest 2012, the technical festival of NIT Trichy
%    \end{itemize}
%
%    \textbf{Electronics} 
%\begin{itemize} \itemsep -2pt
%        \item Designed an accelerometer-gyroscore-magnetometer-based \emph{3D mouse} for controlling 3D CAD objects. The design won \textit{the GE Industrial Defined Problem} at Shaastra 2012, the technical festival of IIT Madras. 
%    \end{itemize}
%
%    \textbf{Community Service} 
%\begin{itemize} \itemsep -2pt
%    \item Volunteer for Association for India Development (AID), a charity
%      organization supporting sustainable development projects in India.
%    \item Volunteered for National Social Service (NSS) at IIT Madras (2009-10). Devised scientific experiments and models and created scientific website content for teaching secondary school students.
%  \end{itemize}

\section{\sc Distinctions}
\begin{itemize} \itemsep -2pt
    \item ECE Graduate Fellowship 2013, Johns Hopkins University
    \item Hamburger Fellowship 2013, Johns Hopkins Univeristy
    \item WISE Scholarship 2012, DAAD, Germany
    \item All India Rank \textbf{191} in {IIT-Joint Entrance Examination (IIT-JEE)} 2009 (among over 400,000 students) 
    \item Awarded Kishore Vaignayik Protsahan Yojana (KVPY) Fellowship 2008 by Dept. of Science and Technology, Govt. of India
    \item Awarded National Talent Search (NTS) Scholarship 2007 by National Council of Education, Research and Training, Govt. of India
    \item Member of IIT Madras team at the National Robotics Contest, Abu Robocon 2011. Placed among the Top 5 in India
    %\item Winner of Autonomous robotics and Image processing robotics at Techfest 2012, NIT Trichy, India
    \item Winner of GE Industrial Defined Problem at Shaastra 2012, IIT Madras, India for design of accelerometer-gyroscope-magnetometer-based 3D-mouse
  \end{itemize}

\section{\sc Skills}

\begin{tabular}{@{}p{0.8in}p{6in}}

Languages:& C/C++, Python, Bash, MATLAB\\
Toolkits: & KALDI, HTK \\

\end{tabular}

\section{\sc References}

Will be provided on request.

\end{resume}
\end{document}
